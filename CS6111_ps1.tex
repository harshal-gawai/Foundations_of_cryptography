\documentclass[11pt]{exam}
\printanswers
\usepackage{amsmath,amssymb,complexity}
\usepackage{datetime,enumerate,palatino}
\usepackage{setspace}

\newcommand{\F}{{\mathbb{F}}}
\newcommand{\Z}{{\mathbb{Z}}}
\newcommand{\N}{\mathbb{N}}
\newcommand{\Q}{\mathbb{Q}}

\begin{document}
\setstretch{1.1}

\hrule
\vspace{3mm}
\noindent
{\sf IITM-CS6111 : Foundations of Cryptography  \hfill Assignment \#1 }
\vspace{3mm} \\
\noindent
{\sf Topic: Perfect Encryption \hfill Due on : $14^{th}$ Dec, 2020}
\vspace{3mm}
\hrule

{\small 
\begin{itemize}
\itemsep 0pt
\item All answers must be written in mathematically rigorous form.
\item Use LaTeX to generate the pdf file of your solution. \textbf{NO} photographs/scanned copies of hand written solutions are accepted. 
\item Students are allowed to form groups (maximum of $3$ members) for this assignment. However, referring to internet etc is strictly disallowed.
\item Submit solutions at moodle link before the deadline one per team. Clearly mention the names and roll numbers of all the teammates involved.
\end{itemize}
}
% uncomment the lines below and fill name and roll number of teammates here

%\hrule
%\vspace{3mm}
%\noindent
%{\sf Roll No: \hfill Full Name: } 
%\vspace{3mm}\\
%\noindent
%{\sf Roll No: \hfill Full Name: } 
%\vspace{3mm}\\
%\noindent
%{\sf Roll No: \hfill Full Name: } 
%\vspace{3mm}
\hrule

\begin{questions}
\question[10] ({\bf \textit{Perfect secrecy}}) 
Suppose a Cryptosystem achieves perfect secrecy for a particular plaintext probability distribution. Prove that perfect secrecy is maintained for any plaintext probability distributions.
%\begin{solution}
%    Uncomment the lines above and below and write your solution here
%\end{solution}

\question[10] ({\bf \textit{Affine cipher}}) 
The \textbf{affine cipher} is a type of monoalphabetic substitution cipher, where each letter in an alphabet is mapped to its numeric equivalent, encrypted using a simple mathematical function, and converted back to a letter.
\begin{parts}
\part \textbf{i)} Prove that Affine cipher achieves perfect secrecy iff keys are used with equal probability 1/312.
\part \textbf{ii)} Compute H(k $\mid$ c) and H(k $\mid$ p,c) for Affine cipher (assume keys are equiprobable and plaintext are equiprobable).
\end{parts}
%\begin{solution}
%    Uncomment the lines above and below and write your solution here
%\end{solution}

\question[10] ({\bf \textit{Shannon's Theorem}}) 
Prove that by redefining the key space, the key-generation algorithm \textbf{\textit{Gen}} chooses a key uniformly at random from the key space, without changing \textbf{Pr[C=c $\mid$ M=m]} for any m, c.
%\begin{solution}
%    Uncomment the lines above and below and write your solution here
%\end{solution}

\question[10] ({\bf \textit{Shannon's Theorem}}) 
Prove that by redefining the key space, \textbf{\textit{Enc}} is deterministic without changing \textbf{Pr[C=c $\mid$ M=m]} for any m, c. 
%\begin{solution}
%    Uncomment the lines above and below and write your solution here
%\end{solution}

\question[10] ({\bf \textit{Shannon's Theorem}}) 
Prove or refute: An encryption scheme with message space \textbf{M} is perfectly secret iff for every probability distribution over \textbf{M} and every $c_0, c_1 \in C$  we have \textbf{ Pr[C= \boldmath $c_0$] = Pr[C= $c_1$]}. 
%\begin{solution}
%    Uncomment the lines above and below and write your solution here
%\end{solution}

\end{questions}
\end{document}


